\section{Empirical Evaluation}
  
In this section we compare the solutions for traffic networks modeled as a QTM
before and after the introduction of a light rail.
%
We consider both fixed-time control, i.e., a non-adaptive control plan, and
optimized adaptive control obtained by solving the
MILP~(\ref{eq:objFunc},~\ref{c:turnProb}--\ref{c:pd:holdTransit}).
%
The obtained solutions are simulated using a microsimulator and their total
travel time and observed delay distribution are used as comparison metrics.
%
Our hypothesis is that the optimized adaptive approach is able to mitigate the
impact of introducing light rail w.r.t.\ both metrics.
%
%In the remainder of this section, we present the traffic networks considered in
%the experiments, our methodology, and the results.






\subsection{Networks}


\begin{figure*}
\centering
\includegraphics[width=0.9\textwidth]{all_in_one.png}%
%
%% FWT: hack to be able to cite each figure individually.
%
\begin{subfigure}{1em}{\phantomcaption\label{fig:net:arterial}}\end{subfigure}%
\begin{subfigure}{1em}{\phantomcaption\label{fig:net:grid}}\end{subfigure}%
\begin{subfigure}{1em}{\phantomcaption\label{fig:transit1}}\end{subfigure}%
\begin{subfigure}{1em}{\phantomcaption\label{fig:transit2}}\end{subfigure}%
\begin{subfigure}{1em}{\phantomcaption\label{fig:demand_profiles}}\end{subfigure}%
%
\vspace{-3mm}
%
\caption{Networks used to evaluate the performance: (a) an arterial road with
parallel light rail; and (b) an urban grid with crisscrossing streets and light
rail.
%
Light rails schedules: (c) \textbf{slow} (long and infrequent) light rail; 
and (d) \textbf{fast} (short and frequent) light rail.
%
(e) weight functions for generating demand profiles.}
%
\vspace{-3mm}
\end{figure*}

We consider two networks of differing complexity: an arterial crossed by four
side streets (\cref{fig:net:arterial}) and a 3-by-3 grid (\cref{fig:net:grid}).
%
The queues receiving cars from outside of the network are
marked in \cref{fig:net:arterial,fig:net:grid} and we refer to them as input queues.
%
The maximum queue capacity~(\QMAX{i}) is 60 vehicles for non-input queues and
infinity for input queues to prevent interruption of the input demand due to
spill back from the stop line. 
%
The free flow speed is 50 km/h and the traversal time of each queue $i$~(\QDELAY{i}) is set at 30s,
except for the output queues on Network 1 where the traversal
time is 10s.
%
For each street, flows are defined from the head of each queue $i$ into the tail
of the next queue $j$;
%
there is no turning traffic ($\FTURN{i}{j}=1$), and the maximum flow rate
between queues, \FMAX{i}{j}, is set at 0.5 vehicles/s.
%
All traffic lights have two phases, north-south and east-west, and for each
traffic light \tl and phase $k$, \PTMIN{\tl}{k} is 10s, \PTMAX{\tl}{k} is 30s,
\CTMIN{\tl} is 20s, and \CTMAX{\tl} is 60s.



%\subsection{Fixed-Time Control Constraints}
%
%We simulate a fixed-time traffic light controller by employing the QTM to
%optimize a fixed phase duration, \PTFIXED{\ell}{k}, over the
%planning horizon.
%%
%We implement this by replacing the bounds constraints on $\pd[n]{\ell}{k}$
%($\pd[n]{\ell}{k} \le \PTMAX{\ell}{k}$ and \ref{c:minPhase}), with fixed duration
%constraints, employing the \textit{big-M} method to apply the constraints only
%while the phase is inactive, where 
%$\PTFIXED{\ell}{k} \in [\PTMIN{\ell}{k}, \PTMAX{\ell}{k}]$.
%%
%\begin{cAlign}
%        \pd{\ell}{k} &\le \PTFIXED{\ell}{k} + \PTMAX{\ell}{k} \p[n]{\ell}{k}
%  \tagconstrain{c:pd:fixedUB}\\
%%
%  \pd{\ell}{k} &\ge \PTFIXED{\ell}{k} - \PTMAX{\ell}{k} \p[n]{\ell}{k}
%  \tagconstrain{c:pd:fixedLB}
%\end{cAlign}
% 
%Constraints \ref{c:pd:fixedLB} and \ref{c:pd:fixedUB} are only applied for time
%intervals $n$ where $\tn[n] > \CTMAX{\tl}$, to allow the controller to select an
%optimized phase offset at the start of each plan.
%%TODO talk about how the fixed phase times fit around transit crossings.
%


\subsection{Experimental Methodology}






We evaluate each network using both fixed-time control and the optimized
adaptive control in two scenarios: before the introduction of light rail and
after.
%
For the latter scenario, we consider the two different light rail schedules:
%
a \textbf{slow light rail} with a crossing duration of 50s, a period of 200s, and a
travel time of 100s between lights (\cref{fig:transit1});
%
and a \textbf{fast light rail} with a crossing duration of 20s, period of 160s, and
travel time of 80s between lights (\cref{fig:transit2}).
%
On Network 2, the North-South schedule is offset by 100s for the slow light rail
and 80s for the fast light rail to avoid a collision at $l_5$.


%
%% TODO(fwt): improve this description and say upfront that the demand levels
%% are parametrized by \alpha
%
Each network is evaluated at increasing demand levels up to the point where
$\inq{i}$ becomes saturated.
%
%% TODO(fwt): we need to clarify that both approaches have access to the whole
%% "future", i.e., perfect information about the incoming cars.
%
For each demand level, traffic is injected into the network in bursts for 600s,
and the number of cars entering the network through $i$ at time $n$ is defined
as $\QIN{i}{n} = \mathrm{max}(\alpha \beta \mathrm{w}_i(\tn[n]),\beta)$ where:
%
$\mathrm{w}_i$ is the weight function in \cref{fig:demand_profiles}
corresponding to the letter label of $i$ in \cref{fig:net:arterial,fig:net:grid};
%
$\beta$ is the maximum inflow rate in vehicles per \DT, as annotated at the
start of queue $i$ in \cref{fig:net:arterial,fig:net:grid}; and
%
$\alpha \in (0,2]$ is the scaling factor for the demand level being evaluated.


For each evaluation, we first generate a signal plan using QTM configured as
either an optimized adaptive controller or a fixed-time controller.
%
We use a problem horizon \TMAX large enough, typically in the range 1000s --
1500s, to allow all traffic to clear the network, that lets us measure the
incurred delay in all the cars.
%
Next, we microsimulate the signal plans on their respective the networks using
the Intelligent Driver Model (IDM)~\cite{treiber2000congested}.
%
For the microsimulation, all vehicles have identical IDM parameters: length
$l=3$~m, desired velocity $v_0 = 50$~km/h, safe time headway $T=1.5$~s, maximum
acceleration $a=2 \text{ m/s}^2$, desired deceleration $b = 3 \text{ m/s}^2$,
acceleration exponent $\delta = 4$, and jam distance $s_0 = 2$~m.
%
We follow the same demand profiles used for computing the signal plans for the
microsimulation.


For all experiments, we used Gurobi as the MILP solver running on a
heterogeneous cluster with 2.8GHz AMD Opteron~4184, 3.1GHz AMD Opteron~4334 (12
cores each), and 2Ghz Intel Xeon E5405 (4 cores). We use 4 cores for each run of
the solver.
%
We limit the MIP gap accuracy to 0.02\% and 0.1\% for the arterial and grid
networks, respectively.
%
Due to Gurobi's stochastic strategies, runtimes for the solver can vary, and we
do not set a time limit.
%
The optimized adaptive solutions are typically found in real time (less than
200s), while fixed-time plans can take significantly longer (on average 4000s
for a 1000s horizon); however, once the fixed-time solution is found, it can be
deployed indefinitely.
%
%\toIain{Do you have the cputimes by any chance in your log files so that we
%could say what is the average?}




\subsection{Results}


\begin{figure*}[t!]
%
\centering
%  trim={<left> <lower> <right> <upper>}
%
%% TODO(fwt): add to the delay plots and boxplots their corresponding number. We
%% need to do this in the figure to save space.
%
\vspace{-5mm}
\begin{subfigure}{0.49\textwidth}
\includegraphics[keepaspectratio,width=\linewidth, trim={0 0cm 0 0cm}]{network1_delay.pdf}
\phantomcaption \label{fig:arterial:delayCurve}
\end{subfigure}
\begin{subfigure}{0.49\textwidth}
\includegraphics[keepaspectratio,width=\linewidth]{network2_delay.pdf}
\phantomcaption \label{fig:grid:delayCurve}
\end{subfigure}
%\vspace{-10mm}
\begin{subfigure}{0.49\textwidth}
%  \footnotesize{(c)} 
\includegraphics[keepaspectratio,width=\linewidth, trim={0 0.7cm 0 1.0cm}]{network1_boxplots.pdf}
\phantomcaption \label{fig:arterial:boxplot}
\end{subfigure}
\begin{subfigure}{0.49\textwidth}
%  \footnotesize{(d)}   
\includegraphics[keepaspectratio,width=\linewidth, trim={0 0.7cm 0 1.0cm}]{network2_boxplots.pdf}
\phantomcaption \label{fig:grid:boxplot}
\end{subfigure}
%\vspace{-5mm}
\begin{subfigure}{0.49\textwidth}
  \centering
  %\footnotesize{(e) Fixed Controller Side Street $q_5$ to $q_6$}
\end{subfigure}
\begin{subfigure}{0.49\textwidth}
  \centering
 % \footnotesize{(f) Fixed Controller Arterial $q_2$ to $q_{10}$}
\end{subfigure}
%
\begin{subfigure}{0.49\textwidth}
\includegraphics[keepaspectratio,width=\linewidth,trim={0 0.9cm 0 0.0cm}]{network1_microsim_fixed_slow_side.pdf}
\phantomcaption \label{fig:microsim1:fixed:side}
\end{subfigure}
\begin{subfigure}{0.49\textwidth}
\includegraphics[keepaspectratio,width=\linewidth,trim={0 0.9cm 0 0.0cm}]{network1_microsim_fixed_slow_arterial.pdf}
\phantomcaption \label{fig:microsim1:fixed:arterial}
\end{subfigure}

%\vspace{-3mm}

\begin{subfigure}{0.49\textwidth}
  \centering
  %\footnotesize{(g) Optimized Controller Side Street $q_5$ to $q_6$}
\end{subfigure}
\begin{subfigure}{0.49\textwidth}
  \centering
  %\footnotesize{(h) Optimized Controller Arterial $q_2$ to $q_{10}$}
\end{subfigure}
\begin{subfigure}{0.49\textwidth}
\includegraphics[keepaspectratio,width=\linewidth,trim={0 0.9cm 0 0.0cm}]{network1_microsim_opt_slow_side.pdf}
\phantomcaption \label{fig:microsim1:adapt:side}
\end{subfigure}
\begin{subfigure}{0.49\textwidth}
\includegraphics[keepaspectratio,width=\linewidth,trim={0 0.9cm 0 0.0cm}]{network1_microsim_opt_slow_arterial.pdf}
\phantomcaption \label{fig:microsim1:adapt:arterial}
\end{subfigure}
%
\caption{Average delay by network demand for the arterial (a) and grid (b)
networks.
%
(c,d): Box plots representing the observed distribution of delay for 3
different values of demand for each network, comparing delay distribution
without the light rail with the impact of the fast and slow schedules.
Optimized adaptive control is able to mitigate the impact of light rail on average delay,
while also producing better signal plans (lower maximum and upper quartile delay).
%
%The mean is presented as a red square in the box plots.
%
%
(e-h): microsimulation time-distance plots of side street $q_5$ to $q_6$ and
arterial $q_2$ to $q_{13}$ of the arterial network with the slow light rail
schedule.  While both controllers can find coordinated policies along the
arterial, the optimized adaptive controller is able to clear out the queues (horizontal
flow lines) in the side streets following the transit of the light rail.
%
%Table (i): improvement in average delay using optimized control compared to
%fixed control with the introduction of light rail, assuming fixed control was
%used pre light rail, and evaluated at the demand level corresponding to point
%$\clubsuit$ in (a) and (b). Optimized control is able to mitigate the impact to
%within 10\% of the pre light rail average delay in each case.
%
%% TODO(fwt): add a better summary of the overall results, i.e., add a ``mini''
%% conclusion here
%
%The fixed-time controller has prioritized the arterial phase duration over
%the side street duration and suffers from accumulative queue build up in the
%side streets following each transit of the light rail.
%%
%The optimized controller is able to clear out the queue build up in the side
%streets by increasing the phase time of the side street for a cycle after
%the transit has passed through, and then returns to a schedule that prioritizes
%the arterial depending on the changes in traffic density.
}
%
\label{fig:delayCurveAndBoxplot}
\label{fig:network1_microsim}
\end{figure*}


\cref{fig:arterial:delayCurve,fig:grid:delayCurve} show, for each network, the
average delay per vehicle as a function of demand for both fixed-time
and optimized adaptive control approaches in three scenarios: before the light
rail and after the installation of light rail using the slow and the fast
schedules.
%
As we hypothesized, optimized adaptive control is able to mitigate the impact of
the introduction of light rail and it marginally increases the average delay
when compared with the average delay produced by fixed-time controller
\textbf{before} the light rail.
%
Moreover, as shown in \cref{fig:arterial:boxplot,fig:grid:boxplot}, the
optimized adaptive controller also produces better signal plans than the
fixed-time controller, i.e., plans with smaller median, third quartile, and
maximum delay.
%
Over all our experiments, the maximum observed delay for the optimized adaptive
signal plans after the introduction of light rail is no more than
%
%% TODO(fwt): The results improved and we can (and have to) provide a tighter
%% bound.
%
the double of the maximum delay before the light rail.
%
%\cref{fig:network_hist} provides more details on the behavior of the signal
%plans for demand level $\rm
%II$~(\cref{fig:arterial:delayCurve,fig:grid:delayCurve}) by showing the
%cumulative number of cars by number of observed stops.
%%
%Note that in \cref{fig:network1_slow:hist:arterial,fig:network1_slow:hist:side}
%the fixed-time controller chooses to prioritize the arterial over the side
%streets, but overall (\cref{fig:network1_slow:hist}) the optimized controller
%does better with less stops at all frequencies.


%\begin{figure*}[t!] \centering
%%  trim={<left> <lower> <right> <upper>}
%%\begin{subfigure}{0.49\textwidth}
%%\includegraphics[keepaspectratio,width=\linewidth]{network1_t1_reduction.pdf}
%%\caption{~}\label{fig:arterial:impact:t1}
%%\end{subfigure}
%\begin{subfigure}{0.49\textwidth}
%\includegraphics[keepaspectratio,width=\linewidth]{network2_t1_reduction.pdf}
%\caption{~}\label{fig:grid:impact:t1}
%\end{subfigure}
%%\begin{subfigure}{0.49\textwidth}
%%\includegraphics[keepaspectratio,width=\linewidth]{network1_t2_reduction.pdf}
%%\caption{~}\label{fig:arterial:impact:t2}
%%\end{subfigure}
%\begin{subfigure}{0.49\textwidth}
%\includegraphics[keepaspectratio,width=\linewidth]{network2_t2_reduction.pdf}
%\caption{~}\label{fig:grid:impact:t2}
%\end{subfigure}
%%
%\vspace{-2mm}
%\caption{Impact on average delay for the arterial (first column) and grid
%(second column) networks for both light rail schedules (rows) in different
%scenarios (curves) of traffic control system before and after installation of
%light rail.
%%%
%The x-axis is the percentage of cars switching to the public transportation
%and the y-axis is the impact after the light rail is installed.
%%
%Negative impact represents increase in average delay.
%%
%The vehicle demand for (a-d) are marked as $\clubsuit$ in their
%respective plots in \cref{fig:delayCurveAndBoxplot}.}
%%
%\label{fig:impact} \end{figure*}


%
%% FWT: Commenting this because we don't have the space to show these plots.
%% Moreover, they require a lot of explanation what is an indication that they
%% should be redone with clarity in mind.
%
%To further illustrate the benefits of using optimized adaptive control to
%minimize the impact of adding light rail to traffic network, \cref{fig:impact}
%shows the impact on average delay as a function of the percentage of cars that
%are switching to the light rail.
%%
%In these plots, the demand level is fixed
%%
%%and its value is labeled by $\clubsuit$ in \cref{fig:delayCurveAndBoxplot},
%%
%and higher values are better~(i.e., there is a decrease in the average delay)
%and zero means that there is no change after installing light rail.
%%
%For the three combinations of before and after policies presented,
%%
%%(we omit the case of using optimized adaptive before the light rail and
%%fixed-time after the light rail),
%%
%we can see that, while keeping the fixed-time controller requires from 7.5\% to
%40\% of the drives to switch to light rail in order to obtain the same average
%delay as before its installation, the optimized adaptive approach requires only
%from 2.5\% to 10\% of the drivers to switch when already using optimized
%adaptive control \textbf{before} the light rail.
%%
%When compared fixed-time before the light rail and optimized adaptive after, the
%average delay stays the same if only the 2\% and 5\% of the drives switch to the
%light rail when considering the slow light rail schedule in the arterial and
%grid networks, respectively;
%%
%moreover, the average delay \textbf{decreases} for both networks using the fast
%light rail schedule even if no driver switches to the newly installed public
%transportation.

%\begin{figure*}[t!] \centering
%%  trim={<left> <lower> <right> <upper>}
%%
%\begin{subfigure}{0.68\textwidth}
%  \centering
%  \footnotesize{Network 1}
%\end{subfigure}
%\begin{subfigure}{0.2\textwidth}
%  \centering
%  \footnotesize{Network 2}
%\end{subfigure}\\[2mm]
%\begin{subfigure}{0.275\textwidth}
%  \centering
%  \qquad \qquad \footnotesize{Arterial}
%\includegraphics[keepaspectratio,width=\linewidth]{network1_hist_slow_arterial.pdf}
%\caption{~}\label{fig:network1_slow:hist:arterial}
%\end{subfigure}
%%
%\begin{subfigure}{0.2\textwidth}
%  \centering
%  \footnotesize{Side Streets}
%\includegraphics[keepaspectratio,width=\linewidth]{network1_hist_slow_side.pdf}
%\caption{~}\label{fig:network1_slow:hist:side}
%\end{subfigure}
%%
%\begin{subfigure}{0.2\textwidth}
%  \centering
%  \footnotesize{Overall}
%\includegraphics[keepaspectratio,width=\linewidth]{network1_hist_slow.pdf}
%\caption{~}\label{fig:network1_slow:hist}
%\end{subfigure}
%%
%\begin{subfigure}{0.2\textwidth}
%  \centering
%  \footnotesize{Overall}
%\includegraphics[keepaspectratio,width=\linewidth]{network2_hist_slow.pdf}
%\caption{~}\label{fig:network2_slow:hist}
%\end{subfigure}
%%
%\begin{subfigure}{0.28\textwidth}
%\includegraphics[keepaspectratio,width=\linewidth]{network1_hist_fast_arterial.pdf}
%\caption{~}\label{fig:nwrwork1_fast:hist:arterial}
%\end{subfigure}
%\begin{subfigure}{0.2\textwidth}
%\includegraphics[keepaspectratio,width=\linewidth]{network1_hist_fast_side.pdf}
%\caption{~}\label{fig:network1_fast:hist:side}
%\end{subfigure}
%\begin{subfigure}{0.2\textwidth}
%\includegraphics[keepaspectratio,width=\linewidth]{network1_hist_fast.pdf}
%\caption{~}\label{fig:network1_fast:hist}
%\end{subfigure}
%\begin{subfigure}{0.2\textwidth}
%\includegraphics[keepaspectratio,width=\linewidth]{network2_hist_fast.pdf}
%\caption{~}\label{fig:network2_fast:hist}
%\end{subfigure}
%\vspace{-2mm}
%\caption{Impact of controller on number of stops. Top row is for the slow light rail schedule. Bottom row is for the fast light rail schedule}
%\label{fig:network_hist}
%\end{figure*}

\newcommand{\cellSymbol}{\ensuremath{\star}\xspace}

\begin{table}
\centering
\footnotesize
\begin{tabular}{p{16mm}|c|r|r|r}
  & \multicolumn{1}{c|}{Light Rail}
  & \multicolumn{1}{c|}{Fixed}
  & \multicolumn{1}{c|}{Opt. Adapt.}
  & \multicolumn{1}{c}{\textbf{Improv.}}
\\ \hline \hline
\multirow{3}{*}{\parbox{16mm}{Arterial ($\clubsuit$) @ 3900 V/s}}
  & No     &  38.2 s & 30.8 s    & \textbf{24.0\%}     \\
  & Slow   &  67.4 s & 39.5 s    & \textbf{70.6\%}     \\
  & Fast   &  47.0 s & \cellSymbol 32.6 s    & \textbf{44.2\%}     \\
\hline
\multirow{3}{*}{\parbox{16mm}{Grid ($\spadesuit$) ~ ~ @ 4800 V/s}}
  & No     &  48.7 s & 45.8 s    & \textbf{6.3\%}      \\
  & Slow   & 100.7 s & 56.4 s    & \textbf{78.6\%}     \\
  & Fast   &  72.3 s & \cellSymbol 47.8 s    & \textbf{30.3\%}     \\
\end{tabular}
\caption{Average delay in seconds and improvement of optimized adaptive
controller over fixed-time controller for both networks and the three light rail
scenarios.
%
The demand level is fixed and correspond to the points $\clubsuit$ and
$\spadesuit$ in \cref{fig:arterial:delayCurve,fig:grid:delayCurve},
respectively.
%
The improvement obtained by our optimized adaptive approach when a light rail is
introduced ranges from 24\% to 78.6\% w.r.t.\ the fixed-time approach.
%
In two cases (marked with \cellSymbol), the average delay obtained by
our approach after the introduction of a light rail is
\textbf{smaller} than the delay of the fixed-time controller with \textbf{no
light rail}.}
\label{fig:table}
\end{table}

%\begin{table}
%\footnotesize
%\centering
%\begin{tabular}{l|r|r|r|r}
%  \multicolumn{1}{c}{}
%  & \multicolumn{2}{c}{Arterial@3900 V/h ($\clubsuit$)}
%  & \multicolumn{2}{c}{Grid@4800 V/h ($\clubsuit$)}
%\footnotesize
%\\ \hline \hline
%%
%Light rail sch.
%%% FWT: not sure why the columns are not aligned in the center of the double
%%% column above. Thus, I'm using ~ as a temporary hack
%  & \multicolumn{1}{c|}{~~~~~slow~~~~~}
%  & \multicolumn{1}{c|}{fast}
%  & \multicolumn{1}{c|}{~~~slow~~~}
%  & \multicolumn{1}{c}{fast}
%\\ \hline
%%
%Fixed        & 65.0\%  &  92.5\%  & 60.0\%  &  82.5\% \\
%Adaptive    & 97.5\%  & 105.0\%  & 90.0\%  & 100.5\% \\ \hline
%\textbf{Improvement}   & \textbf{32.5\%}  &  \textbf{12.5}\%  & \textbf{30.0}\%
%                       &  \textbf{18.0\%}  \\
%\end{tabular}
%\caption{Maximum percentage of the original traffic such that the delay is not
%increase after the introduction of light rail.
%%
%The demand level is fixed and correspond to the point
%$\clubsuit$ in \cref{fig:arterial:delayCurve,fig:grid:delayCurve}.
%%
%The delay obtained by the fixed-time controller \textit{before} the
%introduction of light rail is used as baseline.
%%
%The optimized adaptive controller can successfully mitigate the impact of
%introducing a light rail: the considered networks can be used by 90\% to 105.5\%
%of the original demand without increasing in average delay.}
%%
%\label{fig:table}
%\end{table}

To further illustrate the advantages of the optimized adaptive controller, we show
in \cref{fig:microsim1:fixed:side,fig:microsim1:adapt:side,%
fig:microsim1:fixed:arterial,fig:microsim1:adapt:arterial} the microsimulation
time-distance plots for several streets in Network 1.
%
The y-axis of these plot shows the distance along the street, and the x-axis
shows the evolution over time.
%
Each colored trace represents the journey of a vehicle along the street.
%
Traffic signals at fixed distances down the street appear as black horizontal
dashed lines, where a solid bar represents that the phase is inactive (i.e., the
light is red) and traffic cannot pass; otherwise the phase is active.
%
%Where the line is broken the phase is active, the light is green and cars can
%proceed past.
%
The time-distance plots capture the queueing behaviour of the traffic as each
vehicle decelerates when approaching congested traffic, or red or amber light
ahead.
%
When vehicle is stationary, its trace becomes horizontal.


\cref{fig:microsim1:fixed:side,fig:microsim1:fixed:arterial} show that the
fixed-time controller has prioritized the arterial phase duration over the
side street duration to achieve coordinated ``green corridors'' sized for the
average traffic density in the network.
%
As byproduct, the side street under the fixed-time controller suffers from
accumulative queue build up following each transit of the light rail.
%
In \cref{fig:microsim1:adapt:side,fig:microsim1:adapt:arterial}, we see that the
optimized adaptive controller is able to clear out the queue build up in the
side street by increasing the phase time of the side street for a cycle after
the transit has passed through, and then returns to a schedule that prioritizes
the arterial depending on the changes in traffic density.
%
When the traffic density in the arterial is higher than the side streets then
the optimized adaptive controller will coordinate ``green corridors'' along the
arterial.
%
\cref{fig:table} shows the average delay in seconds of the optimized
adaptive and fixed-time controllers for both networks.
%
In the scenarios with a light rail, the improvement obtained by the optimized
adaptive approach ranges from 24\% to 78.6\% w.r.t.\ the fixed-time approach.
%
We can see that the optimized adaptive controller successfully nullifies the
impact of adding a light rail to the networks since the average delay obtained
by it is approximately the same as the average delay for the fixed-time
controller with no light rail:
%
the average delay increased by at most 8 s, and the average delay decreased in
the two cases marked with \cellSymbol in \cref{fig:table}.

