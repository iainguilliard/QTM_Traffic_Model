\section{The Queue Transmission Model (QTM)}


%% TODO(fwt): We need to say that we handle expected number of cars instead of
%% physical cars, so its fine to have fractions of cars.

A Queue Transmission Model (QTM) is the tuple $(\Qset, \Lset, \vecDT, \MatQIN)$,
where \Qset and \Lset are, respectively, the set of queues and lights;
%
\vecDT is a vector of size \Nn representing the discretization of the problem
horizon $[0,\TMAX]$ and \authorHighlight{the duration in seconds of the $n$-th
time interval is denoted as \DT[n];}
%
%%
%% TODO(fwt): Maybe mention that car that are denied entry could be stored in a
%% infinity capacity queue and eventually be allowed to enter the network. Also
%% point out that this case doesn't happen in our experiments
%%
%% TODO(fwt): Maybe mention that add \QIN{i} can estimate through a model
%% learned from historical data.
%
and \MatQIN is a matrix $|\Qset| \times \TMAX$ in which \QIN{i}{n} represents
the flow of cars requesting to enter queue $i$ from the outside of the network
at time $n$.



A \textbf{traffic light} $\tl \in \Lset$ is defined as the tuple~$(\CTMIN{\tl},
\CTMAX{\tl}, \Pset_\tl, \VecPTMIN{\tl}, \VecPTMAX{\tl})$, where:
%
\begin{itemize*}[label={}]
%
\item $\Pset_\tl$ is the set of phases of $\tl$;
%
\item \CTMIN{\tl} (\CTMAX{\tl}) is the minimum (maximum) allowed cycle time for
\tl; and
%
\item \VecPTMIN{\tl} (\VecPTMAX{\tl}) is a vector of size $|\Pset_\tl|$ and
  \PTMIN{\tl}{k} (\PTMAX{\tl}{k}) is the minimum (maximum) allowed time for
  phase $k \in \Pset_\tl$. 
%
\end{itemize*}


A \textbf{queue} $i \in \Qset$ represents a segment of road that vehicles
traverse at free flow speed; once traversed, the vehicles are vertically stacked
in a stop line queue.
% of maximum capacity \QMAX{i}.
%
Formally, a queue~$i$ is defined by the tuple~$(\QMAX{i}, \QDELAY{i}, \QOUT{i},
\Fvec_i, \Prvec_i, \QPset{i})$ where:
%
\begin{itemize*}[label={}]
%
%% TODO(fwt): clarify that this is the capacity of stop line queue
\item \QMAX{i} is the maximum capacity of $i$;
%
\item \QDELAY{i} is the time required to traverse $i$ and reach the stop line;
%
\item \QOUT{i} represents the maximum traffic flow from $i$ to the outside of
  the modeled network;
%
\item $\Fvec_i$ and $\Prvec_i$ are vectors of size \Qn and their $j$-th entry
  (i.e., \FMAX{i}{j} and \FTURN{i}{j}) represent the maximum flow from queue $i$
  to $j$ and the turn probability from $i$ to $j$ ($\sum_{j \in
  \Qset}\FTURN{i}{j} = 1$), respectively; and
%
\item \QPset{i} denotes the set of traffic light phases controlling the outflow
  of queue $i$.
%
\end{itemize*}


Differently than the CTM \cite{daganzo1994cell,lin2004enhanced}, the
\authorHighlight{QTM does not assume that $\DT[n] = \QDELAY{i}$ for all $n$,
that is, the QTM can represent non-homogeneous time intervals.
%
%%(\cref{subfig:example}).
%
The only requirement over \DT[n] is that no traffic light maximum phase time is
smaller than any \DT[n] since phase changes occur only between time intervals;
formally, $\DT[n] \le \min_{\tl \in \Lset, k \in \Pset_\tl} \PTMAX{\tl}{k}$ for
all $n \in \{1,\dots,\Nn\}$.}
%
% TODO \toIain{Maybe bring forward a small network and any other figure that
% would help illustrate the model and comment about it.}



\subsection{Computing Traffic Flows with QTM}

In this section, we present how to compute traffic flows using QTM and
\authorHighlight{non-homogeneous time intervals \DT[]}.
%
We assume for the remainder of this section that a \emph{valid} control plan for
all traffic lights is fixed and given as parameter;
%
formally, for all $\tl \in \Lset$, $k \in \Pset_\tl$, and interval $n \in
\{1,\dots,N\}$, the binary variable $\p{\tl}{k}$ is known a priori and indicates
if phase $k$ of light \tl is active~(i.e., $\p{\tl}{k} = 1$) or not on interval
$n$.


We represent the problem of finding the maximal flow between
capacity-constrained queues as a Linear Program~(LP) over the following
variables defined for all interval $n \in \{1,\dots,\Nn\}$ and queues $i$ and
$j$:
%
\begin{itemize*}[label={}]
%
\item $\q{i} \in [0,\QMAX{i}]$: traffic volume waiting in the stop line of queue
  $i$ at the beginning of interval~$n$;
%
\item $\inq{i} \in [0,\QIN{i}{n}]$: inflow to the network via queue $i$ during
  interval $n$;
%
\item $\outq{i} \in [0,\QOUT{i}]$: outflow from the network via queue $i$ during
  interval $n$; and
%
\item $\f{i}{j} \in [0,\FMAX{i}{j}]$: flow from queue $i$ into queue $j$ during
  interval $n$.
%
\end{itemize*}



%% FWT: These constraints are not necessary because we defined the domain of the
%% variables in the paragraph above
%\inq{i} &\le \QIN{i}{n} \tag{C1}\label{eq:C1}\\ 
%\outq{i} &\le \QOUT{i} \tag{C2}\label{eq:C2}\\
% \q{i} &\le \QMAX{i} \tag{C9}\label{eq:C9}\\




The maximum traffic flow from queue $i$ to queue $j$ is enforced by
\cref{c:turnProbAndMaxFlow}.
%
The inner inequality of \eqref{c:turnProbAndMaxFlow} ensures that only the
fraction $\FTURN{i}{j}$ of the total internal outflow of $i$ goes to $j$, and
the outer inequality forces the flow from $i$ to $j$ to be zero if all phases
controlling $i$ are inactive (i.e., $\p{\ell}{k} = 0$ for all $k \in
\QPset{i}$).
%
%If more than one phase $\p{\ell}{k}$ is active, then
%\eqref{c:turnProbAndMaxFlow} is subsumed by the domain upper bound of
%$\f{i}{j}$.
%
\begin{cAlign} 
%
\f{i}{j} \le \FTURN{i}{j} \sum_{k=1}^{\Qn} \f{i}{k}
%\tagconstrain{c:turnProb}\\ \f{i}{j}
\le \FMAX{i}{j} \sum_{\p{\ell}{k} \in \QPset{i}} {\p{\ell}{k}}
%\tagconstrain{c:maxFlow}
\tagconstrain{c:turnProbAndMaxFlow}
%
\end{cAlign}




%% TODO(fwt): improve the beginning of this paragraph

To simplify the presentation of remainder of the LP, we define the helper
variables \qin{i}~\eqref{def:qin} and \qout{i}~\eqref{def:qout} to,
respectively, represent the volume of traffic to enter and leave queue $i$
during interval~$n$, and  $\tn[n] = \sum_{x=1}^{n} \DT[x]$ to represent the time
elapsed since the beginning of the problem until the end of interval \DT[n].
%
%
%\tn[n]~\eqref{def:tn}
%
%
\begin{cAlign}
%
\qin{i} &= \DT (\inq{i} + \textstyle \sum_{j=1}^{\Qn} \f{j}{i}) \tagconstrain{def:qin} \\
%
\qout{i} &= \DT (\outq{i} + \textstyle  \sum_{j=1}^{\Qn} \f{i}{j})
\tagconstrain{def:qout}
%\\
%
%\tn[n] &= \sum_{x=1}^{n} \DT[x] \tagconstrain{def:tn}
%
\end{cAlign}



% Introduce the idea of V_i
\authorHighlight{In order to account for the misalignment of the different \DT[]
and \QDELAY{i}, we need to find the volume of traffic that entered queue $i$
between two arbitrary points in time $x$ and $y$ ($x \in [0,\TMAX]$, $y \in
[0,\TMAX]$, and $x < y$), i.e., $x$ and $y$ might not coincide with any $t_n$
for $n \in \{1,\dots,N\}$.
%
This volume of traffic, denoted as $\Vol_i(x,y)$, is obtained by integrating
\qin{i} over $[x,y]$ and is defined in~\eqref{eq:vol} where $m$ and $w$ are the
index of the time intervals s.t. $\tn[m] \le x < \tn[m+1]$ and $\tn[w] \le y <
\tn[w+1]$.
%
Because the QTM dynamics are \emph{piecewise linear}, \qin{i} is a step function
w.r.t.~time and this integral reduces to the sum of \qin{i} over the intervals
contained in $[x,y]$ and the appropriate fraction of \qin[m]{i} and \qin[w]{i}
representing the misaligned beginning and end of $[x,y]$.
%
{\small
\begin{equation}
\Vol_{i}(x,y)\! =\!
  (\tn[m+1]\!-\! x) \frac{\qin[m]{i}}{\DT[m]}\!+\!\left(
  \sum_{k=m+1}^{w-1} \qin[k]{i} \right)\!+\! (y\!-\! \tn[w])
  \frac{\qin[w]{i}}{\DT[w]}
  \label{eq:vol}
\end{equation}}
}


Using these helper variables, \eqref{c:qUpdate} represents the flow conservation
principle for queue $i$ where $\Vol_i(\tn[n-1]-\QDELAY{i},\tn[n]-\QDELAY{i})$ is
the volume of cars that reached the stop line during \DT[n].
%
Since \vecDT and \QDELAY{i} for all queues are known a priori, the indexes $m$
and $w$ used by $V_i$ can be pre-computed in order to encode \eqref{eq:vol};
moreover, \eqref{c:qUpdate} represents a non-first order Markovian update
because the update considers the previous $w-m$ time steps.
%
To ensure that the total volume of traffic traversing $i$ (i.e.,
$\Vol_i(\tn[n] - \QDELAY{i}, \tn[n])$) and waiting at the stop line does not
exceed the capacity of the queue, we apply~\eqref{c:10}.
%
\begin{cAlign}
%
& \q{i} = \q[n\!-\!1]{i} \! - \! \qout[n\!-\!1]{i} \! + \!
\Vol_i(\tn[n\!-\!1] \! - \! \QDELAY{i},\tn[n] \! - \! \QDELAY{i}) \tagconstrain{c:qUpdate}\\
%
& \Vol_i(\tn[n] - \QDELAY{i}, \tn[n]) + \q{i} \le \QMAX{i}
\tagconstrain{c:10}
%
\end{cAlign}


%\begin{figure*}[t!]
%\centering
%%  trim={<left> <lower> <right> <upper>}
%\includegraphics[width=1.0\textwidth,trim={0cm 0cm 0cm 0cm},clip]{convergence_NH.pdf}
%\caption{An example showing the convergence between a homogeneous solution with
%$\Delta t=1.0$ and a non-homogeneous solution over 30 seconds for the same
%network. By using non-homogeneous time steps the same solution is found with
%only 14 sample points compared to 30 for homogeneous solution.}
%\label{fig:converg_c}
%\end{figure*}



As with MILP formulations of CTM (e.g. \trbcite{lin2004enhanced}),
QTM is also susceptible to \emph{withholding traffic}, i.e., the
optimizer might prevent cars from moving from $i$ to $j$ even though the
associated traffic phase is active and $j$ is not full, e.g., this may
reserve space for traffic from an alternate approach that allows the MILP
to minimize delay in the long-term even though it leads to unintuitive traffic
flow behavior.
%
We address this well-known issue through our objective function~\eqref{eq:objFunc} by
maximizing the total outflow~\qout{i}
%
%(i.e., both internal and external outflow)
%
of~$i$ plus the inflow~\inq{i} from the outside of the network to~$i$.
%
This quantity is weighted by the remaining time until the end of the problem
horizon \TMAX to force the optimizer to allow as much traffic volume as possible
into the network and move traffic to the outside of the network as soon as
possible.
%% FWT: not sure if we need to give this extra details
%\trbcite{lin2004enhanced} derive an objective function for the minimisation of
%total delay based on the difference between the cumulative departure and
%arrival curves at the origin and destination. However, such an approach
%requires the network to be cleared at the end of the optimisation period.


\begin{equation}
\max 
 \sum_{n=1}^{\Nn} \sum_{i=1}^{\Qn} (\TMAX - \tn + 1) (\qout{i} + \inq{i})
\tag{O1}\label{eq:objFunc}
\end{equation}


%\begin{figure*}[t!]
%\centering
%%  trim={<left> <lower> <right> <upper>}
%\includegraphics[width=0.55\textwidth,trim={0cm 0cm 0cm 0cm},clip]{cumu_plot_final_6l.pdf}
%\caption{Cumulative arrival (blue) and departure (green) curves, and the
%resulting delay curve (red). The departure curve is maximized by the objective
%function \eqref{eq:objFunc}, which has the same effect as minimizing the area
%under the delay curve.}
%\label{fig:cumu_delay_plot}
%\end{figure*}


The objective \eqref{eq:objFunc} corresponds to minimizing delay in CTM models,
e.g., \eqref{eq:objFunc} is equivalent to the objective function (O3) in
\trbcite{lin2004enhanced} for their parameters $\alpha = \beta = 1$.
%
%
%Figure (fig:cumu-delay-plot) depicts this equivalence using the cumulative
%number of cars entering and leaving a network as a function of time.
%%
%The delay experienced by the vehicles travelling through this network (red curve
%in Figure (fig-cumu-delay-plot) equals the horizontal difference at each point
%between the cumulative departure and arrival curves (less the free flow travel
%time through the network).
%%
%Maximizing \qout{i} weighted by $(\TMAX - \tn +1)$ in \eqref{eq:objFunc} is the
%same as forcing the departure curve to be as close as possible to the arrival
%curve as early as possible; therefore, the area between arrival and departure is
%minimized, which in turn minimizes the delay.



%The objective function~\eqref{eq:objFunc} and
%constraints~(\ref{c:turnProb}--\ref{c:10}) form the LP representing the
%dynamic, piecewise linear model of flow in a QTM network over time when a
%control plan \p{\tl}{k} is given as an input parameter.



%\begin{figure*}[t!]
%\centering
%\subfigure[]{
%\label{subfig:converg_a}
%\includegraphics[width=1.0\textwidth,trim={0cm 0cm 0cm 0cm},clip]{convergence.pdf}
%}
%\subfigure[]{
%\label{subfig:converg_b}
%\includegraphics[width=1.0\textwidth,trim={0cm 0cm 0cm 0cm},clip]{convergence_vari.pdf}
%}
%\label{fig:conv}
%%
%\caption{Approximations of a queue volume obtained using homogeneous
%\DT[] = 1s using: (a) homogeneous \DT[] = 2.5s and 5s; and (b) non-homogeneous
%$\DT[n] \approx 0.0956n + 0.9044$ for $n \in \{1,\dots,17\}$.  Here we see
%that (b) achieves accuracy in the near-term that somewhat degrades over
%the long-term, where accuracy will be less critical for receding horizon control.}
%%% Note: would actually be nice to say something about
%%
%\end{figure*}


%To illustrate the representation tradeoff offered by non-homogeneous time
%intervals, we computed flows and queue volumes for a fixed signal control plan
%derived for homogeneous \mbox{\DT[n] = 1s} (ground truth) using different
%discretizations.
%%
%Figure (subfig:converg-a) shows the approximation of the ground truth using
%homogeneous \DT[] = 2.5 and \DT[] = 5.0, and Figure (subfig:converg-b) using
%non-homogeneous time intervals that linearly increases from 1s to 2.5s, i.e.,
%$\DT[n] \approx 0.0956n + 0.9044$ for $n \in \{1,\dots,17\}$.
%%
%As Figure (subfig:converg-a) shows, large time steps can be rough approximations
%of the ground truth.
%%
%Non-homogeneous discretization (Figure (subfig:converg-b)) exploit this fact to
%provide a good approximation in the initial time steps and progressively
%decrease precision for points far in the future.



\section{Traffic Control with MILP-encoded QTM}

In this section, we show how to compute the optimal control plan by extending
the LP (\ref{eq:objFunc},~\ref{c:turnProbAndMaxFlow}--\ref{c:10}) into an
Mixed-Integer LP (MILP).
%
Formally, for all~$\tl \in \Lset$, $k \in \Pset_\tl$, and interval $n \in
\{1,\dots,N\}$, the phase activation parameter~$\p{\tl}{k} \in \{0,1\}$ becomes
a free variable to be optimized.
%
In order to obtain a valid control plan, we enforce that one phase of traffic
light \tl is always active at any interval~$n$~\eqref{c:onlyOnePhaseOn} and that
phase changes happen sequentially~\eqref{c:seqPhases}
%
%, i.e., if phase~$k$ was
%active during interval~$n-1$ and has become inactive in interval~$n$, then
%phase~$k+1$ must be active in interval~$n$.
%
(\eqref{c:seqPhases} assumes that $k+1$ equals 1 if $k = \Pn{}$).
%
%% TODO: We need to understand this issue better:
% \toIain{I removed the constraint $\p{\ell}{k} + \p{\ell}{k+1} \le 1$ because it
% is subsumed by $\p{\ell}{k} \in \{0,1\}$ and \eqref{c:onlyOnePhaseOn}}
%
%
%% FWT: note sure if the text bellow is necessary
%Note that there could be more than one queue mapped to each $\p[]{\ell}{k}$, or
%their could be none
%
\begin{cAlign}
%
\textstyle \sum_{k=1}^{\Pn} \p{\ell}{k} &= 1\tagconstrain{c:onlyOnePhaseOn}\\
%
\p[n-1]{\ell}{k} &\le \p{\ell}{k} + \p{\ell}{k+1}\tagconstrain{c:seqPhases}
%
\end{cAlign}


%\begin{figure*}[t!]
%\centering
%%  trim={<left> <lower> <right> <upper>}
%\subfigure[]{
%\label{fig:pd:inc}
%\includegraphics[width=0.45\textwidth]{phase_plot_fig_1.pdf}}
%\subfigure[]{
%\label{fig:pd:inactive}
%\includegraphics[width=0.45\textwidth]{phase_plot_fig_2.pdf}}
%\subfigure[]{
%\label{fig:pd:resetAndLB}
%\includegraphics[width=0.45\textwidth]{phase_plot_fig_3.pdf}}
%\subfigure[]{
%\label{fig:cycleTimeC}
%\includegraphics[width=0.45\textwidth]{phase_plot_fig_4.pdf}}
%%
%\caption{Visualization of constraints (\ref{c:pd:incUB}--\ref{c:cycleUB})
%for a traffic light \tl as a function of time.
%%
%(a--c) present, pairwise, the constraints (\ref{c:pd:incUB}--\ref{c:minPhase})
%for phase $k$ (\pd{\ell}{k} as the black line) and the activation variable
%\p[n]{\ell}{k} in the small plot.
%%
%(d) presents the constraints for the cycle time of \tl (\ref{c:cycleLB} and
%\ref{c:cycleUB}), where T.C.T. is the total cycle time and is the left hand side
%of both constraints.
%%
%For this example, $\PTMIN{\ell}{k}=1$, $\PTMAX{\ell}{k}=3$, $\CTMIN{\ell}=7$, and
%$\CTMAX{\ell}=8$.}
%\label{fig:phase_plots}
%\end{figure*}



Next, we enforce the minimum and maximum phase durations (i.e.,
$\PTMIN{\ell}{k}$ and $\PTMAX{\ell}{k}$) for each phase $k \in \Pset_\tl$ of
traffic light \tl.
%
To encode these constraints, we use the helper variable $\pd{\ell}{k} \in
[0,\PTMAX{\ell}{k}]$, defined by constraints
(\ref{c:pd:incUB}--\ref{c:pd:reset}), that:
%
(i) holds the elapsed time since the start of phase $k$ when $\p[n]{\ell}{k}$ is
active~(\ref{c:pd:incUB},\ref{c:pd:incLB});
%
(ii) is constant and holds the duration of the last phase until the next
activation when $\p[n]{\ell}{k}$ is
inactive~(\ref{c:pd:inactiveUB},\ref{c:pd:inactiveLB}); and
%
(iii) is restarted when phase~$k$ changes from inactive to
active~\eqref{c:pd:reset}.
%
Notice that (\ref{c:pd:incUB}--\ref{c:pd:reset}) employs the \textit{big-M}
method to turn the cases that should not be active into subsumed constraints
based on the value of $\p[n]{\tl}{k}$.
%
We use~\PTMAX{\ell}{k} as our large constant since $\pd[n]{\ell}{k} \le
\PTMAX{\ell}{k}$ and $\DT[n] \le \PTMAX{\ell}{k}$.
%
Similarly, \cref{c:minPhase} ensures the minimum phase time of $k$ and is
not enforced while $k$ is still active.
%
%Figures (fig:pd:inc,fig:pd:inactive,fig:pd:resetAndLB) present an example of how
%(\ref{c:pd:incUB}--\ref{c:minPhase}) work together as a function of the time $n$ 
%for $\pd[n]{\ell}{k}$; the domain constraint $0 \le \pd[n]{\ell}{k} \le
%\PTMAX{\ell}{k}$ for all $n \in \{1,\dots,\Nn\}$ is omitted for clarity.
%
%% FWT: I think the mathematical definition of \pd is redundant now.
%\begin{equation}
%\pd{\ell}{k} = 
%\begin{cases}
%\pd[n-1]{\ell}{k} + \DT[n-1] & \p[n-1]{\ell}{k}=1,\p{\ell}{k}=1\\
%\pd[n-1]{\ell}{k} & \p{\ell}{k}=0\\
%0 & \p[n-1]{\ell}{k}=0,\p{\ell}{k}=1
%\end{cases}
%\label{def:pd}
%\end{equation}
%
%% TODO(fwt): I believe that that \pd{\ell}{k} (above) could be redefined by
%% shifting it by delta_n resulting in a d_{l,k,n} representing the total time
%% until the end of interval **n** instead of n-1. This would be more intuitive
%% and would simplify c:cycleLB and c:cycleUB. This is what I propose:
%% d_{l,k,n} = d_{l,k,n-1} + delta-t_{n}  if p_{l,k,n-1} = p_{l,k,n} = 1
%%           = d_{l,k,n-1}                if p_{l,k,n} = 0
%%           = delta-t_{n}                if p_{l,k,n-1} 0 and p_{l,k,n} = 1
%
%% FWT: this constraint is define as the variable domain
%\pd{\ell}{k} &\le \PTMAX{\ell}{k}\tag{C19}\label{eq:C19}\\
%
\begin{cAlign}
%
\pd{\ell}{k} &\le
  \pd[n-1]{\ell}{k} + \DT[n-1] \p[n-1]{\ell}{k} 
  + \PTMAX{\ell}{k} (1 - \p[n-1]{\ell}{k})\tagconstrain{c:pd:incUB}\\
%
\pd{\ell}{k} &\ge
  \pd[n-1]{\ell}{k} + \DT[n-1] \p[n-1]{\ell}{k}
  - \PTMAX{\ell}{k} (1 - \p[n-1]{\ell}{k})\tagconstrain{c:pd:incLB}\\
%
\pd{\ell}{k} &\le \pd[n-1]{\ell}{k} + \PTMAX{\ell}{k} \p[n-1]{\ell}{k}
  \tagconstrain{c:pd:inactiveUB}\\
\pd{\ell}{k} &\ge \pd[n-1]{\ell}{k} - \PTMAX{\ell}{k} \p{\ell}{k}
  \tagconstrain{c:pd:inactiveLB}\\
%\end{cAlign}
%%
%\begin{cAlign}
%%
\pd{\ell}{k} &\le \PTMAX{\ell}{k}(1 - \p{\ell}{k} + \p[n-1]{\ell}{k})
  \tagconstrain{c:pd:reset}\\
%
\pd{\ell}{k} &\ge \PTMIN{\ell}{k}(1 - \p{\ell}{k}) \tagconstrain{c:minPhase}
%
\end{cAlign}
%


Lastly, we constrain the sum of all the phase durations for light \tl to be
within the cycle time limits \CTMIN{\tl}~\eqref{c:cycleLB} and
\CTMAX{\tl}~\eqref{c:cycleUB}.
%
In both \eqref{c:cycleLB} and \eqref{c:cycleUB}, we use the duration of phase 1
of \tl from the previous interval $n-1$ instead of the current interval $n$
because \eqref{c:pd:reset} forces \pd[n]{\tl}{1} to be 0 at the beginning of
each cycle;
%
however, from the previous end of phase 1 until $n-1$, \pd[n-1]{\tl}{1} holds
the correct elapse time of phase 1.
%
Additionally, \eqref{c:cycleLB} is enforced right after the end of the each
cycle, i.e., when its first phase is changed from inactive to active.
%
%%The value \eqref{c:cycleLB} and \eqref{c:cycleUB} over time for a traffic light
%%\tl is illustrated in Figure (fig:cycleTimeC).
%
\begin{cAlign}
%
\pd[n-1]{\ell}{1} + \sum\limits_{k=2}^{\Pn} \pd{\ell}{k} &\ge \CTMIN{\ell}
(\p{k}{1} - \p[n-1]{k}{1}) \tagconstrain{c:cycleLB}\\
%
\pd[n-1]{\ell}{1} + \sum\limits_{k=2}^{\Pn} \pd{\ell}{k} &\le \CTMAX{\ell}
\tagconstrain{c:cycleUB}
%
\end{cAlign}
%
The MILP~(\ref{eq:objFunc},~\ref{c:turnProbAndMaxFlow}--\ref{c:cycleUB}) encodes
the problem of finding the optimal traffic control plan in a QTM network without
light rail.

\paragraph{Light Rail Constraints}
%
To incorporate a fixed-schedule light rail in our model, we post-process our
MILP model by fixing the free variable $\p[n]{\tl}{k}$ for all $n$ s.t.\ the
light rail uses phase $k$ of \tl at time $n$.
%
Formally, given a schedule $S_{\tl}(k,n) \in \{0,1\}$ where~$1$ represents that
the light rail uses phase $k$ of \tl at time $n$, we replace
(\ref{c:pd:incUB}--\ref{c:minPhase}) by \eqref{c:forceTransit} and
\eqref{c:pd:holdTransit} when $\sum_{k} S_{\tl}(k,n) > 0$.
%
\begin{cAlign}
%
\p[n]{\ell}{k} &= S_{\ell}(k,n)\tagconstrain{c:forceTransit}\\
%
\pd{\ell}{k} &= \pd[n-1]{\ell}{k}\tagconstrain{c:pd:holdTransit}
%
\end{cAlign}


%To incorporate a fixed-schedule light rail in our model, we enforce phase $k$ used
%by the light rail is always on when the transit crosses at light \tl, i.e., we add
%the constraint $\p[n]{\ell}{k} = 1$ for all $n$ s.t.\ the light rail uses phase $k$
%of \tl at time $n$.
%%
%%
%We also hold $\pd{\ell}{k}$ constant while the light rail crosses \tl, i.e., 
%$\pd{\ell}{k} = \pd[n-1]{\ell}{k}$.
%%
%Since the light rail schedule is fixed and known a priori, adding these extra
%constraints are done in a preprocessing phase before the MILP is solved.
